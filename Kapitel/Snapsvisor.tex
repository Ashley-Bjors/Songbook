
\fakesection{Snapsvisor}

\fancypagestyle{Snapsvisor}{
\fancyhead{} % clear all header fields
\fancyhead[LE,RO]{\textbf{Snapsvisor}}
}
\pagestyle{Snapsvisor}



\subsection{\textbf{Om snapsen}}


Om man nu dricker snaps så är det fullständigt uteslutet att peta i
sig några droppar utan att sjunga en sång därtill.
Snapsarnas historia är lång och bristfälligt dokumenterad. Genom
muntlig tradition har dock en del kunskaper bevarats till
eftervärlden.

I den grå forntiden var snapsarnas antal i princip oändligt. Under
senare delen av 1800-talet krympte antalet till maximalt 17. Dessa
känner vi fortfarande namnen på: 

\begin{multicols}{2}[]
\begin{enumerate}
    \item Helan
    \item Halvan
    \item Tersen
    \item Quarten
    \item Quinten
    \item Sexten
    \item Septen
    \item Rivan
    \item Rafflan
    \item Rännan
    \item Smuttan
    \item Smuttans unge
    \item Lilla Manasse
    \item Lilla Manasses brorsa
    \item Femton dropppar
    \item Kreaturens återuppståndelse
    \item Ett evigt liv    
\end{enumerate}
\end{multicols}

Alla sjutton dricks endast undantagsvis och då av personer som
inte har alltför stora ambitioner för resten av aftonen.

I modern tid är det vanligt att man låter sig sjunga några fler visor
genom att inte tömma snapsglaset varje gång. Den något ryska
seden är dock att helan till varje pris ska tömmas i botten.

Traditionen är att alltid börja sjunga Helan!

\subsection{\textbf{Feta små ryskor}}

Mel: Marsche militaire av Franz Schubert\bigskip

Feta små ryskor som svettas om fötterna,

trampar potatis som sedan skall jäsas till sprit.

Transpirationen viktig e’

ty den ger fin bouquet.

Vårtor och svampar följer me’,

men vad gör väl de’?\bigskip

För...

Vi vill ha sprit, vill ha sprit, vill ha mera sprit

även om följderna blir att vi får lida skit.

Flickor: Flaskan och glaset gått i sin

Pojkar: Hit med vin, BRÄNNEVIN!

Flickor: Tror ni att vi är fyllesvin?

Pojkar: JA! (Fast större)

\subsection{\textbf{Helan går}}

Helan går

Sjung hoppfadderallan-lallan-lej

Helan går

sjung hoppfaddreallan-lej.

Och den som inte Helan tar

han heller inte halvan får,

Helan gååååår…\bigskip

Sjung hoppfadderallan-lej

\subsection{\textbf{Helan var bra }}

Mel: Å j’änta å ja\bigskip


Helan var bra, nu ska vi ta

halvan i detta rycket.

En går väl an, men två är minsann

inte ett dugg för mycket.\bigskip

Ensam är stark, två river mer

så skynda er nu att slänga den ner

kanske ni kan få mer om en stund,

ja skål på er allihopa. 

\subsection{\textbf{Hur länge skall på borden}}

Mel: Jag minns den ljuva tiden\bigskip


Hur länge ska på borden

den lilla halvan stå.

Ska snart ej höras orden

nu halvan går låt gå.

Det ärvda vikingasinne

till supen trår igen

och helans trogna minne

i halvan går igen. 

\subsection{\textbf{Hej feskegubbar!}}

Mel: Hej tomtegubbar\bigskip

Hej feskegubbar, dra i näten

och hala hem några dunkar!

Hej feskegubbar, ej förgäten

att taga duktiga klunkar!

Nu Helan går, vi mer förmår.

Vänd botten upp och sen gutår!

Hej feskegubbar, ej förgäten

att taga duktiga klunkar!

\subsection{\textbf{Humlorna}}

Sjungs om drinken "Geting" dricks

Mel: Här kommer Karl-Alfred boy\bigskip

Vi äro små humlor vi, bzz-bzz,

vi äro små humlor vi, bzz-bzz.

Vi äro små humlor som tar oss en geting,

vi äro små humlor vi, bzz-bzz!

\subsection{\textbf{Denna thaft}}

Mel: Helan\bigskip

Denna thaft

är den bätha thaft thythemet haft.

Denna thaft

är den bätha thaft dom haft.

Och den thom inte har nån kraft,

han dricka thkall av denna thaft.

Denna thaft, till landth, till sjöth, till havth.

\subsection{\textbf{Nu tar vi den}}

Mel: O Tannenbaum\bigskip

$\:\|$ Nu tar vi den $\:\|$\bigskip

\subsection{\textbf{Elarens snapsvisa}}

Mel: Så lunka vi så småningom

Go’ vänner gör som broder Seth,

låt nubben sista resan ta..

Ty minns att varje liten skvätt

gör elar’n god och gla’.\bigskip

$\:\|$ Upp på kårhuset är det sed

att låta helan snabbt gå ned,

ta sig se’n dito en,

dito två, dito tre.

Det gör var elare! $\:\|$\bigskip

\subsection{\textbf{Törsten rasar}}

Mel: Längtan till landet\bigskip


Törsten rasar uti våra strupar.

Tungan hänger torr och styv och stel.

Men snart vankas stora, långa supar

var och en får sin beskärda del.

Snapsen kommer, den vi vilja tömma,

denna nektar likt olympens saft

kommer oss att våra sorger glömma

snapsen skänker hälsa, liv och kraft.\bigskip

Helan tänder helig eld i själen

halvan rosar livet som en sky.

Tersen känns fån hjässan ner i hälen

kvarten gör en som en mänska ny.

Låt oss skåla med varann go vänner,

skål för våran levnads glada hopp. 


\subsection{\textbf{Finsk snapsvisa (kort)}}

NU!!

\subsection{\textbf{Finsk snapsvisa (lång)}}

Int nu..\bigskip

Men NU!!

\subsection{\textbf{Snöret}}

Text: 

Mel: Hej tomtegubbar\bigskip

Tänk om jag hade lilla nubben uppå ett snöre i halsen!

Tänk om jag hade lilla nubben uppå ett snöre i halsen!

Och kunde dra den upp och ner,

så att det kändes som många fler!

Tänk om jag hade lilla nubben uppå ett snöre i halsen!

\subsection{\textbf{Öppna bäsken}}

Mel: Öppna landskap\bigskip

Jag trivs bäst bland öppna flaskor,

känna doften av en Bäsk.

Som jag lagrat några månader,

då den blir som allra bäst.

För jag bränner ju mitt brännvin själv,

och kryddar det med malörtsblom.

Mitt sinne lyfts av dryckens krav,

där den glimmar i sitt glas.

\subsection{\textbf{Skogskokarvisa}}

Mel: Mors lilla Olle\bigskip

Mors lilla Olle i skogen gick

leta’ bland furorna, fast han ej fick.

Sågade sju och en halv

och drog hem,

kokade O.P. och cognac av dem.

Mor nu fick syn på’n

gav till ett vrål:

Kokar du skogen till ren alkohol?

- Klart, sade Olle, och allt som jag fällt

är till vår fest här på gården beställt.

\subsection{\textbf{Ack’va visa}}

Mel: Vi går över daggstänkta berg\bigskip

Vi går över ån efter sprit, fallera,
men efter vatten går vi ej en bit, fallera.
Ja drick kära broder fast näsan är röder
ty tids nog så blir den ack’ va vit, fallera.

\subsection{\textbf{Ångbåten}}

Mel: Jazzgossen\bigskip


Och så kommer det en ångbåt

som säger tuut- tuut- tuut,

och så kommer det en ubåt

som säger...

(Varpå snapsen sveps , och gurglas innan den sväljs)

\subsection{\textbf{Ode till flygbensinet}}

Mel: Helan går\bigskip

Flygbensin är drycken för vart fyllesvin

Flygbensin är rått som terpentin

Det piggar upp din trötta kropp

och hettar upp ditt blodomlopp

Flygbensin ...

är Mannfreds medicin.

\subsection{\textbf{Tallen}}

Mel: Längtan till landet\bigskip

Fordom odlade man vindruvsranka

av vars saft man gjorde ädelt vin.

Nu man pressar saften av en planka,

doftande av äkta terpentin.

Höj nu bägaren, o broder, syster

och låt svenska skogen rinna kall

ner i strupen och, om du är dyster,

låt oss dricka upp en liten tall.

\subsection{\textbf{En liten fyllhund}}

Mel: Mors lilla Olle\bigskip

En liten fyllhund på krogen satt,

rosor på kinden - men blicken var matt,

Läpparna små, liksom näsan var blå.

"Ack, om jag kunde så skulle jag gå."

\subsection{\textbf{Nu tar vi rus}}

Mel: Nu är det ljus, här i vårt hus\bigskip

Nu tar vi rus här i vårt hus

drick tills du faller och fall ner igen.

Baren är din,

tag dig en gin,

spy och försvinn.

Mamma, pappa, alla fatta glasen,

ställa er som jag till fylleasen.

Låt nu spriten

och akvaviten

få mera sällskap

- ge hit en!

\subsection{\textbf{Mera brännvin}}

Mel: Internationalen\bigskip


Mera brännvin i glasen,

mera glas på vårt bord,

mera bord på kalasen,

mer kalas uppå vår jord.

Mera jordar kring månen,

mera månar kring mars,

mera marscher till Skåne,

mera Skåne, gubevars!

\subsection{\textbf{Nubben Goa}}

Mel: Gubben Noa\bigskip

Nubben Goa, Nubben Goa,

är en heders sup.

Ut i Alko-hålet,

töm den om du tål’et.

Nubben Goa, Nubben Goa,

är en heders sup.

\subsection{\textbf{Ser du stjärnan i det blå}}

Mel: Ser du stjärnan i det blå\bigskip

$\:\|$ Ser du stjärnan i det blå?

Ta en sup så ser du två.

Tar du sedan något mer,

så ser du fler. $\:\|$\bigskip

\subsection{\textbf{Byssan lull}}

Mel: Byssan lull\bigskip

$\:\|$ Byssan lull utav brännvin blir man full,

och slipsen man doppar i smöret. $\:\|$

Ja, näsan den blir röd,

och ögonen får glöd,

och tusan så bra blir humöret.

\subsection{\textbf{Måsen}}

Mel: Måsen\bigskip


Det satt en mås på en klyvarbom

och tom i magen var kräke’.

Och skepparns strupe var torr och tom,

när skutan låg där i bläke.

”Jag vill ha sill”, hördes måsen rope’,

och skepparn svarte: ”Jag vill ha OP,

om blott jag får, om blott jag får”!

\subsection{\textbf{Förtsa snapsen heter göken}}

Mel: Räven raskar över isen\bigskip


Första snapsen heter göken.

Första snapsen heter göken.

Får jag lov, får jag lov,

Att byta byxor med fröken?

Andra snapsen den var värre.

Andra snapsen den var värre.

Får jag lov, får jag lov,

Att byta byxor med min herre?

Mina byxor är himmelsblå,

Men med dina är det si och så.

Så, får jag lov, får jag lov,

Att byta byxor med göken?

\subsection{\textbf{Helangorakatt}}

Mel: Vi gå över daggstänkta berg\bigskip

Det var en gång en halangorakatt, Fallera

Som älskade en vanlig bonakatt, Fallera

Och följden blev en jamare

Och den var inte tamare

Den kallas för HALVANGÅR…a-katt, Fallera.

\subsection{\textbf{Korta Sola’}}

[EDITORS NOTE: Hittar inte denna melodi...]

Mel: Sola’\bigskip

Solen den går upp och ner.

Och snapsen den går ner.

\subsection{\textbf{Påfyllningssång}}

Mel: Mors lilla Olle\bigskip

Helan så ensam i magen gick,

Undrade varför ej sällskap han fick?

Värden, ack säg var är halvan i kväll?

Bed honom komma till oss är du snäll.\bigskip

Värden ser glasen ger upp ett skri.

Skynda sig sedan att fort fylla i.

Nu har den kommit till rätta vår vän.

Svep den nu innan den smiter igen.

\subsection{\textbf{Sill och Nubbe}}

Mel: Vi gå över daggstänkta berg\bigskip

Till nubben så tager man sill, fallera

Men också en ansjovis om man vill, fallera

Men om man är oviss

Om sillen är ansjovis

Så tar man bara några nubbar till, fallera.

\subsection{\textbf{En gång i måna’n}}

Mel: Mors lilla Olle\bigskip

En gång i måna’n är månen full,

men aldrig jag sett honom ramla omkull.

Stum av beundran hur mycket han tål,

höjer jag glaset och utbringar skål.

Höjen nu glasen och dricken ur.

Nu, kära bröder, står kvarten i tur.

Nubben, den giver oss ny energi.

Säkert den minskar vårt livs entropi.

\subsection{\textbf{Byssan lull v2}}

Mel: Byssan lull\bigskip

$\|\:$ Bysan lull snart så blir du så full;

Då får du sju jamare på festen. $\:\|$

Den första får du nu,

Den andra får du sen,

De sista får du smyga fram ur västen.

$\|\:$ Byssan lull gå på sittning och bli full,

för oss har det blivit en vana. $\:\|$

Den första var en bäsk,

Den andra var en bäsk,

Den tredje var bäskast av alla.

\subsection{\textbf{Låt tersen gå}}

Mel: I sommarens soliga dagar\bigskip

Med sång skall man hålla ett gill-e

Så där så det ekar i natten den still-e.

Man sjunger till supen och sill-en,

Sätt igång – låt tersen gå –

Hallå, hallå!

Du som är ung, var med och sjung

Sitt inte tyst och trög och tung.

Man ska va’ gla’, tra la la la

Med sång och snaps så går det så bra

Trots rymdsatelliterna rusa.

Låt tersen gå – låt gå –

Hallå, hallå!

\subsection{\textbf{Imse vimse hutt}}

Mel: Imse vimse spindel\bigskip

Imse vimse blir man

Av en liten hutt

Pulsen börjar öka

Hjärtat tar ett skutt

Knäna skälver, näsan blir blå

-fast det är så läskigt

vågar vi ändå.

\subsection{\textbf{Nubben goa v2}}

Mel: Gubben noa\bigskip


Nubben goa, nubben goa

Är en hedersdryck.

När den går till magen,

Blir man lätt i tagen.

Nubben goa, nubben goa

Tar man med en knyck.\bigskip

Nubben goa, nubben goa

Är en hederssup.

Uti alko-hålet

Töm den, om du tål’et.

Nubben goa, nubben goa…\bigskip

(När sången sjungits en gång kan det efterfrågas "Omstart fast på --")


--engelska

Nubben goa, nubben goa

Is a famous drink

Said the Prince of Wales

Ever since det gales:

Nubben goa, nubben goa

Is a famous drink.\bigskip


--Tyska

Nubben goa, nubben goa

Ist ein Ehrenschnaps.

Wann das Glas wir lehren,

will es sich vermehren.

Nubben goa, nubben goa

Ist ein Ehrenschnaps\bigskip

--Latin

Nubben goa, nubben goa

Est in pocula.

Nunc es den bibendum,

hux.flux capiendum.

Nubben goa, nubben goa

Tempus est att ta.\bigskip

\subsection{\textbf{Other lands...}}

Mel: Längtan till landet\bigskip

Other lands have vineyards without number

and their fruit becomes the native wine.

But the Swedes squeeze liquor out of lumber,

smelling fragrantly of terpentine.

Now if you can take another measure,

cool and clear go gurgling down the line.

Let's enjoy the Swedish forest pleasure.

Let's fill up and drink another pine!

\subsection{\textbf{När gäddorna}}

När gäddorna leker i vikar och vass

Och solen går ner bakom Sjöbloms dass.

Ja, då är det vår…

(se upp där nere – nu kommer den!)

Sjung hopp fallerallan lej.

\subsection{\textbf{Jag var tall en gång}}

Mel: Flottarkärlek\bigskip

Jag var tall en gång för länge sen

Med grön och härlig färg

Nu så står jag här i glaset klar och kall.

I alla munnar, alla strupar

Kommer helan som en vän

Den har doften av skogar, sjöar och av berg.

Haderian, hadera, riktigt trevligt skall vi ha.

Lyft på armen, böj på huvudet och svälj.

\subsection{\textbf{Nuskaviklämmasepten}}

Mel: Nu ska vi skörda linet\bigskip

Nuskaviklämmaseptengutår

klämmaentrudeluttomdetgår

TjosanMuhammedsnartärdetvår

julaftonärenfreda

klunkklunkklunkklunkklunkklunkblandaågeblandaåge

Abrakadabraklunk

julaftonärenfreda

\subsection{\textbf{Jag tror, jag tror}}

Mel: Jag tror, jag tror på sommaren \bigskip

Jag tror, jag tror på akvavit

jag tror, jag tror på dynamit

den ger en kraft att sjunga ut

och inga krämpor blir akut.

Man glömmer vardagslivets jäkt

och känner stundens ruseffekt.

En snaps, en skål, en trudelutt

och sen så tar vi våran hutt.


\subsection{\textbf{Låtnamn}}

Text:

Mel: \bigskip

\newpage