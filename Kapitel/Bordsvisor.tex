
\fakesection{Bordsvisor}

\fancypagestyle{Bordsvisor}{
\fancyhead{} % clear all header fields
\fancyhead[LE,RO]{\textbf{Bordsvisor}}
}
\pagestyle{Bordsvisor}



\subsection{\textbf{En liten blå förgätmigej}}

Sjungs knästående, med höger knä i backen och vänster knä
blottat, som tack till Kårpersonalen efter avslutad sittning

Text \& Mel: Ulla Billquist\bigskip

Hur gärna ville jag ej vara

en liten blå förgätmigej,

en liten blå förgätmigej.

Då skulle jag för dig förklara,

hur innerligt jag älskar dig.

(och dig, och dig...)

\subsection{\textbf{Jag har aldrig vatt på snusen }}

Text: 

Mel: O, hur saligt att få vandra \bigskip


Jag har aldrig vart på snusen,

aldrig rökat en cigarr, halleluja!

Mina dygder äro tusen,

inga syndiga laster har jag.

Jag har aldrig sett nått naket,

inte ens ett litet nyfött barn.

Mina blickar går mot taket

därmed undgår jag frestarens garn.

Halleluja . . .\bigskip

Bacchus spelar på gitarren,

Satan spelar på sitt handklaver.

Alla djävlar dansa tango,

säg, vad kan man önska sig mer?

Jo, att alla bäckar vore brännvin,

hela svartån full av bayerskt öl,

konjak i var rännsten

och punsch i varendaste pöl.

Och mera öl . . . 

\subsection{\textbf{Fader Abraham 
}}

Fader Abraham,

Fader Abraham

fyra söner hade Abraham

och dom åt och drack

och dom drack och åt

och dom ropade de så här:\bigskip

Höger arm!

Vänster arm!

Höger fot!

Vänster fot!

Huvudet!

Tungan!

SKÅL! 

\subsection{\textbf{Brännvin är gott}}

Mel: Här kommer Karl-Alfred boy\bigskip

Brännvin är jävligt gott,

brännvin är jävligt gott.

Men, slår man i golvet

så där mellan tolv-ett

då slår man sig jävligt hårt!


\subsection{\textbf{Studielånet}}

Mel: Hej tometegubbar\bigskip

$\|\:$Mitt lilla lån det räcker inte,

det går till öl och till brännvin! $\:\|$

Till öl och brännvin går det åt

och till små flickor emellanåt.

Mitt lilla lån det räcker inte,

det går till öl och till brännvin!

\subsection{\textbf{Kalmarevisan}}

$\|\:$ För uti Kalmare stad

ja där finns det ingen kvast $\:\|$

förrän lördagen.

Hej dick

Hej dack

Jag slog i

och vi drack

Hej dickom dickom dack

hej dickom dickom dack.

För uti Kalmare stad

ja där finns det ingen kvast

förrän lördagen.\bigskip

$\|\:$ När som bonden kommer hem

kommer bondekvinnan ut $\:\|$

och är stor i sin trut

Hej dick . . .\bigskip

$\|\:$ Var är pengarna du fått ?

Jo, dom har jag supit opp ! $\:\|$

Uppå Kalmare slott.

Hej dick . . .\bigskip

$\|\:$ Jag skall mäla dig an

för vår kronbefallningsman $\:\|$

Och du skall få skam

Hej dick . . .\bigskip

$\|\:$ Kronbefallningsmannen vår

satt på krogen i går $\:\|$

Och var full som ett får.

Hej dick . . . \bigskip


$\|\:$ Va’ sa’ bonnen ha te’ mat?

Sura sillar och potat $\:\|$

Det blir sillsallat.

Hej dick . . .\bigskip

$\|\:$ Säg var är din labbrapport?

Ja den har jag supit bort $\:\|$

För den var för kort.

Hej dick . . . 

\subsection{\textbf{Knall och fall}}

[EDITORS NOTE: Hittar varken melodin eller vart låter komm ifrån...]

Text: 

Mel: Mylord\bigskip


Jag är en teknolog

Som härom kvällen låg

I egna spyor som jag hade kastat opp.

Min bästa partyfrack

Som hade gjort mig black

Vid nästa Valborg blir den balens stora 
flopp.\bigskip

Nu har jag Overall

Som mycket mera tål.

Jag dricker öl och spiller ner från topp 
till tå.

Men man blir korpulent,

Trots att man är student.

Fast mycket bättre kan man faktiskt inte 
må.\bigskip

Ut i min nakenhet

Jag inte längre vet.

Jag missar tentor och har inga studielån.

Vill ju bli ingengör,

Men vet ej hur man gör.

Nu har jag givit upp och satsar nog på rån?!

-VILKET FÅN! \bigskip

\subsection{\textbf{Lambo}}

En (eller flera) utnämnd person ställer sig upp med en enhet i handen

För nu glaset till din mun!

Tjo-fa-de-rittan lambo!

Och drick ur, din fylle-hund!

Tjo-fa-de-rittan lambo!

Se, hur dropparna i glaset (Nu dricker den utnämnde upp allt i enheten)

rinner ner i fylle-aset.

Lambo-Hej! Lambo-Hej!

Tjo-fa-de-rittan lambo!\bigskip

(Är personen inte klar ännu börja sjung "Gråa hår")\bigskip

(sjungs av den utnämnde)

Jag nu glaset druckit har,

Tjo-fa-de-rittan lambo!

Ej en droppe finnes kvar,

Tjo-fa-de-rittan lambo!

Som bevis jag nu skall vända

glaset på dess rätta ända.

Lambo-Hej! Lambo-Hej!

Tjo-fa-de-rittan lambo!\bigskip

(Sjungs av resterande)

Ja, han kunde konsten,

han var en riktig fylle-fylle hund.

Låt oss gå till nästa man

och se vad han förmår. 

\subsection{\textbf{Gråa hår }}

$\|\:$ Nu har vi väntat länge nog, länge nog, länge nog $\:\|$\bigskip

Nu börjar vi få gråa hår ...\bigskip

Nu börjar vi få ATP ...\bigskip

Nu får vi ringa Fonus snart...\bigskip

Nu sitter vi på gravens rand... \bigskip

Nu börjar det att lukta lik...\bigskip

Nu bäres liken ut på bår...\bigskip

Nu krypa vi i kistan in…\bigskip

Nu börjar de att skyffla jord…\bigskip

Nu växer mossan på vår grav...\bigskip

Nu har vi blivit till skelett…\bigskip

Nu knackar vi på himlens port…\bigskip

Nu sitter vi hos sankte Per…\bigskip

Nu åker vi i Himlen in…\bigskip

Nu får vi höra harpmusik…\bigskip

Då flyr vi ner i Helvetet…\bigskip

Nu capsar vi med Satan själv, Satan själv,
Satan själv.

Nu capsar vi med Satan själv, i Helvetet.\bigskip

Nu har vi inga verser kvar... 

\subsection{\textbf{Hyfsvisa}}

Kors i all sin dar,

har du brännvin kvar?

Är du sparsam eller snål?

SKÅL!

\emph{Sparsamma sjunger: "Snål!"}

\subsection{\textbf{Rönnerdahl}}

Mel: Sjösala vals\bigskip

Rönnerdahl han skuttar med ett skratt ur sin säng

fastnar i ett lakan, slår näsan i sänggaveln.

Rullar ned på golvet i en våghalsig sväng,

slutar sen att skratta när hans stortå får däng.

Virrig i sin hjärna han reser sig på knä,

och se så mången stjärna fast morgon de nu é,

och se så många blåmärken som redan slagit ut på benen,

Blåa och vackra i morgonens svaga ljus.\bigskip

Rönnerdahl han vinglar upppå osäkra ben,

Och den vita skjortan den slafsar kring vaderna,

Packad som en alika i majsolen sken

Skrålar han för ekorren som gungar på gren.

”Titta” ropar ungarna, ”pappa han är full!”

han raglar runt i stugan där han faller omkull,

och se så många burkar han redan har slängt ut på ängen.

Löwenbrau, Heineken, Faxe och Norrlandsguld.

\subsection{\textbf{Det var länge sen}}

Mel: Det va Längesen man plocka några blommor \bigskip

Det var länge sen jag plocka’ några blommor.

Det var länge sen jag tog några poäng.

Det var länge sen jag handla’ på systemet.

Det var länge sen jag fick en tjej i säng.\bigskip

Men å andra sidan bränner jag ju hemma,

och klarar kärleken alldeles för mig själv.

Det var länge sen jag plocka’ några tentor

men å andra sidan går de om igen. 

\subsection{\textbf{Sittningsvisa}}

Mel: Raska Fötter\bigskip


Raska pojkar dricka fort,fort, fort.

Hela flaskan gick som smort, smort, smort.

Många backar bär vi in.

Många gubbar bär vi ut.

Det är bara roligt.\bigskip

Alla bara ropar öl, öl, öl.

Snälla söta inget söl, söl, söl.

Pelle får en öl så stor.

Punsch får lille, lille bror.

Stina får en Zingo.\bigskip

Snart är goda ölen slut, slut, slut.

Fylleaset bäres ut, ut, ut.

Men till nästa gång igen,

Kommer han, vår gamle vän, ty det har han lovat.\

\subsection{\textbf{I koma}}

Mel: Tre Pepparkaksgubbar\bigskip

I koma, i koma

Vi faller efter hand.

Så röda, så röda

blir ögonen ibland.

I glasen, i glasen

vi finner ljuvlig saft

och ta mig tusan får vi inte

både liv och kraft. 

\subsection{\textbf{Vem kan raggla}}

Mel: Vem kan segla\bigskip

Vem kan raggla för utan vin,

vem är nykter om våren,

vem kan skilja på kron och gin,

utan att smaka på 'ren.\bigskip

Jag kan raggla för utan vin,

å visst var jag nykter den våren,

men jag kan ej skilja på kron och gin,

efter den elfte tåren.\bigskip

Jag berusar mig varje kväll,

jag kan klara en sjuttis,

men ej gå innan festens slut,

utan att fälla tårar.\bigskip

Jag är full nästan året om,

jag kan ej längre tänka,

men jag mår trots allt jävligt bra,

utom när jag är bakfull.\bigskip

Jag kan snart inte längre se,

jag har snart ingen lever,

men inte fan deppar jag för det,

höj nu glasen och skåla! \bigskip

\subsection{\textbf{Skål för våra vänner }}

Mel: \bigskip

Vi skålar för våra vänner
och dom som vi känner
och de som vi inte känner
dom sätter vi på!!!

[[EDITORS NOTE: Bytte ut "dom skiter vi i !!!" till "dom sätter vi på!"]]

\subsection{\textbf{The basic song}}

Varje rad fungerar som en rad kod i basic.
Varje gång låter starts om brukar takten öka

Mel: Mors lilla olle\bigskip

10 LET oss nu fatta i våra glas

20 INPUT en klunk i utav det som där has

30 IF du fått nog THEN till 50 min vän

40 ELSE GOTO-baka till 10 igen

50 END

\subsection{\textbf{Temperaturen}}

Temperaturen är hög uti kroppen

närmare 40 än 37,5.

Ja, så ska det vara när ångan är oppe.

Och så är fallet uti detta nu!\bigskip

Vi rulla, vi rulla....\bigskip

Livet är kort som en barnunges tröja,

ingenting kan vi väl gör åt det.

Men med vår sång kan bekymren vi röja,

bort ifrån vardagens gråaktighet.\bigskip

Vi rulla, vi rulla.... \bigskip

\subsection{\textbf{Fredmans sång nr 21 (kort)}}

Text: Carl Michael Bellman\bigskip



Så lunka vi så småningom

från Bacchi buller och tumult,

när döden ropar; Granne kom,

ditt timglas är nu fullt.

Du gubbe fäll din krycka ner,

och du yngling, lyd min lag,

den skönsta nymf som mot dig ler

inunder armen tag.\bigskip

Ref:

Tycker du att graven är för djup,

nå välan, så tag dig då en sup,

tag dig sen dito en, dito två, dito tre,

så dör du nöjdare.\bigskip

Säg är du nöjd, min granne säg,

så prisa världen nu till slut;

om vi ha en och samma väg,

så följoms åt; drick ut.

Men först med vinet rött och vitt

för vår värdinna bugom oss,

och halkom sen i graven fritt,

vid aftonstjärnans bloss.\bigskip

Ref... \bigskip

\subsection{\textbf{Låtnamn}}

Text: 

Mel: \bigskip

\newpage